\documentclass[11pt, a4paper]{article}
%\usepackage{geometry}
\usepackage[inner=1.5cm,outer=1.5cm,top=2.5cm,bottom=2.5cm]{geometry}
\pagestyle{empty}
\usepackage{graphicx}

\usepackage[usenames,dvipsnames]{color}
\definecolor{darkblue}{rgb}{0,0,.6}
\definecolor{darkred}{rgb}{.7,0,0}
\definecolor{darkgreen}{rgb}{0,.6,0}
\definecolor{red}{rgb}{.98,0,0}
\usepackage[colorlinks,pagebackref,pdfusetitle,urlcolor=darkblue,citecolor=darkblue,linkcolor=darkred,bookmarksnumbered,plainpages=false]{hyperref}
\renewcommand{\thefootnote}{\fnsymbol{footnote}}



\thispagestyle{plain}

%%%%%%%%%%%% LISTING %%%
\usepackage{listings}
\usepackage{caption}
\DeclareCaptionFont{white}{\color{white}}
\DeclareCaptionFormat{listing}{\colorbox{gray}{\parbox{\textwidth}{#1#2#3}}}
\captionsetup[lstlisting]{format=listing,labelfont=white,textfont=white}
\usepackage{verbatim} % used to display code
\usepackage{fancyvrb}
\usepackage{acronym}
\usepackage{amsthm}
\VerbatimFootnotes % Required, otherwise verbatim does not work in footnotes!



\definecolor{OliveGreen}{cmyk}{0.64,0,0.95,0.40}
\definecolor{CadetBlue}{cmyk}{0.62,0.57,0.23,0}
\definecolor{lightlightgray}{gray}{0.93}



\lstset{
%language=bash,                          % Code langugage
basicstyle=\ttfamily,                   % Code font, Examples: \footnotesize, \ttfamily
keywordstyle=\color{OliveGreen},        % Keywords font ('*' = uppercase)
commentstyle=\color{gray},              % Comments font
numbers=left,                           % Line nums position
numberstyle=\tiny,                      % Line-numbers fonts
stepnumber=1,                           % Step between two line-numbers
numbersep=5pt,                          % How far are line-numbers from code
backgroundcolor=\color{lightlightgray}, % Choose background color
frame=none,                             % A frame around the code
tabsize=2,                              % Default tab size
captionpos=t,                           % Caption-position = bottom
breaklines=true,                        % Automatic line breaking?
breakatwhitespace=false,                % Automatic breaks only at whitespace?
showspaces=false,                       % Dont make spaces visible
showtabs=false,                         % Dont make tabls visible
columns=flexible,                       % Column format
morekeywords={__global__, __device__},  % CUDA specific keywords
}

%%%%%%%%%%%%%%%%%%%%%%%%%%%%%%%%%%%%
\begin{document}
\begin{center}
  {\Large \textsc{Comprehensive Exam Practice Questions}}
  MGMT 737
\end{center}
\begin{center}
Spring 2021
\end{center}


In answering these questions, full marks are given for explanations,
not just right answers. Good luck!

\begin{enumerate}
\item In Angrist, Imbens and Rubin (1996), they study as an
  application the effect of military service on civilian
  mortality. The relevant variables are:
  \begin{itemize}
  \item $Z_{i}$: binary variable that person $i$ received a low draft lottery number (such
    that they more likely to be drafted)
  \item $D_{i}$: binary indicator that person $i$ served in the military
  \item $Y_{i}$: binary indicator that person $i$ died between 1974 and 1983 given lottery
  \end{itemize}
  \begin{enumerate}
    \item Write these variables in potential outcome notation. Describe, in words, what each potential outcome means. (Hint: there should be 4 total)
    \item Under what assumptions would a regression of $Y_{i}$ on $D_{i}$ yield the causal effect of military service on mortality rates? Write out this estimand using potential outcomes.
    \item Define the exclusion restriction for draft lottery numbers in terms of potential outcomes.
    \item List an example violation of this restriction.
    \item Under what assumptions would a regression of $Y_{I}$ on $Z_{i}$ yield the causal effect of draft lottery numbers on mortality rates? Write out this estimand using potential outcomes.
    \item Finally, under what additional assumptions could you use an
      instrumental variables approach with the draft lottery numbers
      to identify the effect of military service on mortality? Be
      precise in defining your assumptions in terms of potential
      outcomes.
    \item Imagine that the exclusion restriction is violated. Under
      what settings would this not cause significant bias in the IV
      estimator?
    \end{enumerate}
  \item Lee (2008) considers the impact of a Democrat winning on subsquent victory using a regression discontinuitiy design
  \begin{itemize}
  \item $Z_{i}$: running variable -- vote share margin of victory (RD at $Z_{i} = 0$)
  \item $D_{i}$: winning election
  \item $Y_{i}$: subseuquent victory in an election
    \begin{enumerate}
    \item Write out the estimand for the RD above
    \item Describe a way in which this design could be violated
    \item How would you estimate this effect? Describe the estimation
      procedure (not just what function you would use, but how it
      would be implemented. You do not need to be precise
      mathematically).
    \item A graduate student colleague of yours suggests running a
      linear regresion on both sides of the regression, using the full
      dataset, and then taking the predicted value at the cutoff for
      each regression. What issues might you have with that? Feel free
      to draw a picture.
    \item What issues arise with discrete running variables? How would
      you solve them?
    \end{enumerate}
  \end{itemize}
  \item Consider a random sample of individuals $i= 1, \ldots, n$,
    with treatment status $D_{i}$ and outcome $Y_{i}$.
    \begin{enumerate}
    \item Write out the individual treatment effects estimands using the potential outcome notation.
    \item Write the DAG for this effect
    \item Now imagine that $i= 1, \ldots, n$ instead indexes pairs of
      roommates in a college dorm, with $Y_{i} = (Y_{i1}, Y_{i2})$. If
      we thought treatments from roommates had spillover effects, how
      would you write the potential outcome? Define the different
      potential estimands you could construct using this notation
      (Hint: there should be 4).
    \item Write out the DAG for this setup.
    \item Would a regression of outcomes on the number of people in a
      room who are treated ($X_{i} = D_{i1} + D_{i2}$) capture any of
      these effects? Explain.
    \end{enumerate}
  \item Consider the effect of going to college $D_{i}$ on earnings
    $Y_{i}$.
    \begin{enumerate}
    \item We are given a number of covariates, $X_{i}$, and told that
      conditional on $X_{i}$, strict ignorability holds for $D_{i}$
      and the outcome. Write out what that means in potential outcome notation.
    \item Write down a DAG where this holds.
    \item How would you implement a p-score procedure to estimate the
      average treamtent effect of $D_{i}$ on $Y_{i}$, using the strict
      ignorability condition?
    \item You're now given data on occupation for these individuals
      ($W_{i}$). We might expect that occupation causes changes in
      earnings, and the choice of occupation is causally shifted by
      the decision to go to college. Add this variable to the DAG.
    \item What would happen to our causal estimate if we now added
      $W_{i}$ as a control to our estimation procedure?
    \end{enumerate}
  \item We consider the roll-out of a COVID-19 lockdown across some
    states, but not others. For $t \geq 0$, $D_{it} = 1$ for states in
    the treatment group, and $D_{it} = 0$ for the control states. For
    $t < 0$, $D_{it} = 0$ for everyone. We're interested in the effect
    on economic activity $Y_{i}$, and will use difference-in-differences.
    \begin{enumerate}
    \item Write out a simple parametric model that would let you
      identify the effect of this policy, while not necessarily
      assuming random assignment of $D_{i}$, given two time
      perods. What assumption is necessary?
    \item Write out the regression for how you would estimate a single
      treatment effect in the post-period for the treatment. (you may have done this in the last problem)
    \item What if $D_{it}$ had been randomly assigned? What could you do instead?
    \item Describe a test you can do to test the validity of this
      model. What do you need? What are some potential issues with
      this test?
    \item Now, you get access to policies that have been implemented
      at different times (a staggered roll-out). Describe in words how
      having a staggered roll-out provides a more robust
      identification approach.
    \item What issues might arise if you ran the same regression as in
      part b) above with the staggered roll-outs?  Describe in words,
      or with algebra, or with a graph (or all three). 
    \end{enumerate}
\end{enumerate}
  
\end{document}