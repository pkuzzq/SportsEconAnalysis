\documentclass[11pt, a4paper]{article}
%\usepackage{geometry}
\usepackage[inner=1.5cm,outer=1.5cm,top=2.5cm,bottom=2.5cm]{geometry}
\pagestyle{empty}
\usepackage{graphicx}
\usepackage{fancyhdr, lastpage, bbding, pmboxdraw}
\usepackage[usenames,dvipsnames]{color}
\definecolor{darkblue}{rgb}{0,0,.6}
\definecolor{darkred}{rgb}{.7,0,0}
\definecolor{darkgreen}{rgb}{0,.6,0}
\definecolor{red}{rgb}{.98,0,0}
\usepackage[colorlinks,pagebackref,pdfusetitle,urlcolor=darkblue,citecolor=darkblue,linkcolor=darkred,bookmarksnumbered,plainpages=false]{hyperref}
\renewcommand{\thefootnote}{\fnsymbol{footnote}}

\pagestyle{fancyplain}
\fancyhf{}
\lhead{ \fancyplain{}{Course Name} }
%\chead{ \fancyplain{}{} }
\rhead{ \fancyplain{}{\today} }
%\rfoot{\fancyplain{}{page \thepage\ of \pageref{LastPage}}}
\fancyfoot[RO, LE] {page \thepage\ of \pageref{LastPage} }
\thispagestyle{plain}

%%%%%%%%%%%% LISTING %%%
\usepackage{listings}
\usepackage{caption}
\DeclareCaptionFont{white}{\color{white}}
\DeclareCaptionFormat{listing}{\colorbox{gray}{\parbox{\textwidth}{#1#2#3}}}
\captionsetup[lstlisting]{format=listing,labelfont=white,textfont=white}
\usepackage{verbatim} % used to display code
\usepackage{fancyvrb}
\usepackage{acronym}
\usepackage{amsthm}
\VerbatimFootnotes % Required, otherwise verbatim does not work in footnotes!



\definecolor{OliveGreen}{cmyk}{0.64,0,0.95,0.40}
\definecolor{CadetBlue}{cmyk}{0.62,0.57,0.23,0}
\definecolor{lightlightgray}{gray}{0.93}



\lstset{
%language=bash,                          % Code langugage
basicstyle=\ttfamily,                   % Code font, Examples: \footnotesize, \ttfamily
keywordstyle=\color{OliveGreen},        % Keywords font ('*' = uppercase)
commentstyle=\color{gray},              % Comments font
numbers=left,                           % Line nums position
numberstyle=\tiny,                      % Line-numbers fonts
stepnumber=1,                           % Step between two line-numbers
numbersep=5pt,                          % How far are line-numbers from code
backgroundcolor=\color{lightlightgray}, % Choose background color
frame=none,                             % A frame around the code
tabsize=2,                              % Default tab size
captionpos=t,                           % Caption-position = bottom
breaklines=true,                        % Automatic line breaking?
breakatwhitespace=false,                % Automatic breaks only at whitespace?
showspaces=false,                       % Dont make spaces visible
showtabs=false,                         % Dont make tabls visible
columns=flexible,                       % Column format
morekeywords={__global__, __device__},  % CUDA specific keywords
}

%%%%%%%%%%%%%%%%%%%%%%%%%%%%%%%%%%%%
\begin{document}
\begin{center}
{\Large \textsc{MGMT 737: Applied Empirical Methods}}
\end{center}
\begin{center}
Spring 2021
\end{center}
%\date{September 26, 2014}

\begin{center}
\rule{6in}{0.4pt}
\begin{minipage}[t]{.75\textwidth}
\begin{tabular}{llcccll}
\textbf{Instructor:} & Paul Goldsmith-Pinkham & & &  & \textbf{Time:} & TTh, 1-2:20pm  \\
\textbf{Email:} &  \href{mailto:paul.goldsmith-pinkham@yale.edu}{paul.goldsmith-pinkham@yale.edu} & & & & \textbf{Place:} & Online
\end{tabular}
\end{minipage}
\rule{6in}{0.4pt}
\end{center}
\vspace{.5cm}
\setlength{\unitlength}{1in}
\renewcommand{\arraystretch}{2}


\noindent\textbf{Course Pages:} \begin{enumerate}
\item \url{https://github.com/paulgp/applied-methods-phd}
\item \url{https://yale.instructure.com/courses/64286} [Yale students only]
\end{enumerate}

\vskip.15in
\noindent\textbf{Office Hours:} After class, or by appointment. 

\noindent\textbf{Course Description:} This course is primarily designed for
graduate students interested in econometric methods used in empirical
research. The goal of this class is to provide an overview of
different empirical methods, with an emphasis on practical
implementation.  I will provide a set of lecture slides and
notes. There are additional background papers that are largely
optional.

More generally, this is a course where I focus on providing my
understanding and intuition of empirical methods, as they are used by
practicioners. This means that this is not a course where we will
spend a lot of time on the formal details (beyond what is necessary),
but instead focus on the intuitive framework that guides these
papers. I'll also do my best to communciate how any of these topics
fit together.

This is a course very much focused on communication and
artisanship. By the end of the term, my hope is for three things:

\begin{enumerate}
\item You will have been exposed to a wide range of empirical methods,
  and have at least a passing familiarity with their pros and
  cons. Moreover, you will know where to go look if you decide to use
  these methods.
\item Much of the terminology and jargon that we use in econometric
  methods will be less intimidating to you. When someone says ``I use
  semiparametric inference,'' now instead of intimidate you, it will
  bother you that they are not using clearer language.
\item You will approach research papers with the desire to disentangle
  the underlying framework and ``experiment'' that drives their causal
  inferences.
\end{enumerate}

\noindent\textbf{Assignments:}
There will be problem sets every week. These will involve both
theoretical calculations and computer exercises in which you will be
asked to analyze data sets. You can use any computer package you wish
to use. Solutions will be handed out written in R. Since there will be
a fair number of problem sets, and in order to allow me to post the
solutions quickly on the webpage for the course, I will not accept
late problem sets. If you anticipate difficulty meeting the deadline,
you can ask me for the problem set earlier to give you additional time
to work on it.

You can work together on the problem sets and discuss them with
classmates, but you need to write up the results individually and hand
them in separately. Grades will be based on the problem sets, divided
evenly over the problem sets.

I expect these assignments to be coded from ``scratch.'' I will
specify when canned packages are appropriate. In other words, when
estimating a regression, I am not looking for the results of
\texttt{lm( y $\sim$ x)}. Rather, I expect you to construct two
matrices and calculate the estimates using this. I also expect you to
attempt to maintain good coding practices while doing so -- this will
likely be challenging for those of you who are inexperienced at
programming, so please plan accordingly -- I will not be providing
additional instruction on coding beyond what I cover in class.  See
the following resources in R for guidance (Many thanks to Max Kasy for organizing these materials):

\begin{itemize}
\item Introduction to Base R: \url{https://cran.r-project.org/doc/manuals/r-release/R-intro.pdf}
\item R for Data Science: \url{https://r4ds.had.co.nz/}
\item Guidance on Data Visualization: \url{https://socviz.co/}
\end{itemize}

\vskip.15in
\noindent\textbf{Main References:} %\footnotemark
This is a partial list of various interesting and useful books that will be touched during the course. 
\begin{itemize}
\item Joshua Angrist and J\"orn-Steffen Pischke, {\textit{Mostly Harmless Econometrics}}
\item Scott Cunningham, {\textit{Causal Inference: The Mixtape}},  \url{https://mixtape.scunning.com/}
\item Peter M. Aronow and Benjamin T. Miller, {\textit{Foundations of Agnostic Statistics}}
\item Kieran Healy, {\textit{Data Visualization: A Practical Introduction}}, \url{https://socviz.co/}
\end{itemize} 

\vskip.15in
\noindent\textbf{Prerequisites:}
ECON 550, ECON 551

\bigskip

\textbf{Course Outline}
\begin{enumerate}
\item \underline{Causality, Statistics, and Economics}
  \begin{enumerate}
  \item Potential Outcomes and Directed Acyclic Graphs
    \begin{itemize}
    \item Chapter 7, Aronow and Miller 
    \item Chapter 3, Cunningham
    \item ``Structural vs. Reduced Form'' Language, Confusion and Models in Empirical Economics, Haile \url{http://www.econ.yale.edu/~pah29/intro.pdf} (or on course website)
    \item ``Statistics and Causal Inference'', Holland 1986
    \item ``The Identification Zoo: Meanings of Identification in Econometrics'', Lewbel, 2019
    \end{itemize}
  \item Research Design, Randomization, and Design-Based Inference
    \begin{itemize}
    \item ``Causality and design-based inference'', Bowers, J. \& Leavitt, T. (2020)
    \item ``Instruments, Randomization, and
      Learning about Development'' Deaton (2010)
    \item ``Better LATE Than Nothing: Some Comments on Deaton (2009) and Heckman and Urzua (2009)'' Imbens (2010)
    \item ``Building Bridges between Structural and Program Evaluation Approaches to Evaluating Policy''
    \item ``Let's take the Con out of Econometrics'', Leamer (1983)
    \item ``The Credibility Revolution in Empirical Economics: How Better Research Design is Taking the Con out of Econometrics'', Angrist and Pischke (2010)      
    \end{itemize}
  \item Propensity Scores
    \begin{itemize}
    \item ``The central role of the propensity score in observational
      studies for causal effects'' Rosenbaum and Rubin (1983)
    \item ``Matching As An Econometric Evaluation Estimator'', Heckman, Ichimura, and Todd (1998)
    \item ``Efficient Estimation of Average Treatment Effects Using the Estimated Propensity Score'', Hirano, Imbens and Rider (2003)
    \item ``Propensity Score-Matching Methods for Nonexperimental Causal Studies'', Dehijia and Wahba (2002)
    \item ``Does matching overcome LaLonde's critique of nonexperimental estimators?'', Smith and Todd (2005)
    \item ``Practical propensity score matching: a reply to Smith and Todd'', Dehijia (2005)
    \item ``Nonparametric Estimation of Average Treatment Effects Under Exogeneity: A Review'', Imbens (2004)
%    \item ``Why Propensity Scores Should Not Be Used for Matching.'' King and Nielsen (2019)
    \end{itemize}
  \item Interference, Spillovers and Dynamics
    \begin{itemize}
    \item ``Identification of Endogenous Social Effects: The Reflection Problem'' Manski (1993)
    \item ``Social Networks and the Identification of Peer Effects'' Goldsmith-Pinkham and Imbens (2013)
    \item ``Identification of treatment response with social interactions'' Manski (2013)
    \item ``Estimating average causal effects under general interference.'' Aronow, P. M. \& Samii, C. (2017)
    \item ``Exact p-Values for Network Interference'', Athey, Eckles and Imbens (2018)
    \item ``Estimating peer effects in networks with peer encouragement designs'' Eckles, Kizilcec and Bakshy (2016)
    \item ``Causal Inference under Temporal and Spatial Interference.'' Wang (2020) \url{https://www.yewang-polisci.com/publications}
    \end{itemize}
  \end{enumerate}
\item \underline{Linear Regression}
  \begin{enumerate}
  \item Inference
    \begin{itemize}
    \item ``Robust Standard Errors in Small Samples: Some Practical Advice.'' Imbens and Koles\'ar  (2016)
    \item ``GMM estimation with cross sectional dependence'' Conley (1999)
    \item ``The Standard Errors of Persistence'' Kelly (2019)
    \item ``Clustering, spatial correlations, and randomization inference.'' Barrios et al. (2012)
    \item  ``Sampling-based vs. Design-based Uncertainty in Regression Analysis.'' Abadie et al. (2019)
    \item Abadie et al. ``When Should You Adjust Standard Errors for Clustering?'' 2017
    \item Athey et al. ``Using Wasserstein Generative Adversarial Networks for the Design of Monte Carlo Simulations''
    \end{itemize}
  \item Seimparametrics and Visualization
    \begin{itemize}
    \item ``On Binscatter'' Cattaneo et al. (2019)
    \item ``Validation of Visual Statistical Inference, Applied to Linear Models'', Majumder, Hoffman and Cook (2014)
    \item ``Visual Inference and Graphical Representation in Regression Discontinuity Designs,'' Kortin, Lieberman, Matsudaira, and Shen (2020)
    \item ``Better Data Visualizations: A Guide for Scholars, Researchers, and Wonks'' Schwabish (2021)
    \item ``Data Visualization: A Practical Introduction'', \url{https://socviz.co/}, Healy
    \end{itemize}
  \item Quantile Regression
    \begin{itemize}
    \item Koenker and Hallock. Quantile Regression. 2001
    \item Koenker. Quantile Regression: 40 Years On. 2017
    \end{itemize}
  \item Penalized linear regression
    \begin{itemize}
    \item ``Regression shrinkage and selection via the lasso'' Tibshiriani (1996)
    \item ``The adaptive lasso and its oracle properties'', Zou (2006)
    \item ``Sparse estimators and the oracle property, or the return of Hodges' estimator'' Leeb and Potscher (2008)
    \item ``Preconditioning the lasso for sign consistency'' Jia and Rohe (2015)
    \item ``Machine learning: an applied econometric approach'' Mullainathan and Spiess (2017)
    \item ``Double/debiased/neyman machine learning of treatment effects'' Chernozhukov et al. (2017)
    \item ``Double/debiased machine learning for treatment and structural parameters'' Chernozhukov et al. (2018)
    \item ``Generic machine learning inference on heterogenous treatment effects in randomized experiments'' Chernozhukov et al. (2020)      
    \item ``High-Dimensional Methods and Inference on Structural and Treatment Effects'', Belloni, Chernozhukov  and Hansen (2014)
    \item ``Inference on treatment effects after selection among high-dimensional controls'' Belloni, Chernozhukov  and Hansen (2014)
    \item ``On model selection consistency of Lasso'' Zhao and Yu (2006)
    \item  ``Omitted variable bias of Lasso-based inference methods: A finite sample analysis'' Wuthrich and Zhu (2020)
    \end{itemize}
  \end{enumerate}
\item \underline{Likelihood Methods}
  \begin{enumerate}
  \item Binardy Discrete Choice, GLM and Computational Methods
    \begin{itemize}
    \item ``Discrete Choice Methods with Simulation'' Train (2009) \url{https://eml.berkeley.edu/books/choice2.html}
    \end{itemize}
  \item Multiple Discrete Choices
    \begin{itemize}
    \item ``Maximum score estimation of the stochastic utility model of choice'', Manski (1975)
    \item ``Analysis of covariance with qualitative data'' Chamberlain (1980)
    \item ``Binary Response Models for Panel Data: Identification and Information'' Chamberlain (2010)
    \end{itemize}
  \item Duration models
    \begin{itemize}
    \item ``Econometric Methods for the Duration of Unemployment'', Lancaster (1979)
    \item ``Generalised residuals and heterogeneous duration models: With applications to the Weilbull model'', Lancaster (1985)
    \item ``Duration Dependence and Labor Market Conditions: Evidence from a Field Experiment'', Kroft, Lange and Notowidigdo (2013)
    \item ``Economic duration data and hazard functions'', Kiefer (1988)
    \item ``Duration Models: Specification, Identification and Multiple Durations'', Van Den Berg (2001)
    \end{itemize}
  \item Hierarchical modeling + Bayesian Shrinkage
    \begin{itemize}
    \item ``The Impacts of Neighborhoods on Intergenerational Mobility II: County-Level Estimates'' Chetty and Hendren (2018)
    \item ``Understanding the average impact of microcredit expansions: A bayesian hierarchical analysis of seven randomized experiments'' Meager (2019)
    \item ``Investing for the Long Run when Returns Are Predictable.'' Barberis (2000)
    \end{itemize}
  \end{enumerate}
\item  \underline{Canonical Research Designs}
  \begin{enumerate}
  \item Difference-in-differences
    \begin{itemize}
    \item ``Difference-in-differences with variation in treatment timing'', Goodman-Bacon (2018)
    \item ``Two-way fixed effects estimators with heterogeneous treatment effects'' de Chaisemartin and d'Haultfoeuille (2020)
    \item ``Design-based analysis in difference-in-differences settings with staggered adoption'' Athey and Imbens (2018)
    \item ``Difference-in-differences with multiple time periods'', Callaway and Santa'Anna (2020)
    \item ``Pre-event trends in the panel event-study design'', Freyaldenhoven et al. (2019)
    \item ``On the Use of Two-Way Fixed Effects Regression Models for Causal Inference with Panel Data'', Imai and Kim (2020)
    \item ``Fuzzy differences-in-differences'' de Chaisemartin and d'Haultfoeuille (2018)
    \item ``Semiparametric difference-in-differences estimators'' Abadie (2005)
    \end{itemize}
  \item Event Studies, Synthetic Control + Synthetic DinD
    \begin{itemize}
    \item ``Using synthetic controls: Feasibility, data requirements, and methodological aspects'' Abadie (2019)
    \item ``Synthetic control methods for comparative case studies: Estimating the effect of California’s tobacco control program'' Abadie, Diamond and Hainmueller (2010)
    \item ``Synthetic Difference In Differences'', Arkhangelsky et al. (2019)
    \end{itemize}
  \item Instrumental Variables (Part I)
    \begin{itemize}
    \item ``Identification and estimation of local average treatment effects'' Imbens and Angrist (1994)
    \item ``Identification of causal effects using instrumental variables'' Angrist, Imbens and Rubin (1996)
    \item ``Identification of Causal Effects Using Instrumental Variables: Comment'', Heckman (1996)
    \item ``Instrumental Variables: A Study of Implicit Behavioral Assumptions Used in Making Program Evaluations'' Heckman (1997)
    \item ``Comment on James J. Heckman,`Instrumental Variables: A Study of Implicit Behavioral Assumptions Used in Making Program Evaluations''' Angrist and Imbens (1999)
    \item ``Instrumental variables: response to Angrist and Imbens'' Heckman (1999)
    \end{itemize}
  \item Instrumental Variables (Part II)
    \begin{itemize}
    \item ``On the structure of IV estimands'' Andrews (2019)
    \item ``Weak instruments in instrumental variables regression: Theory and practice'' Andrews, Stock and Sun (2019)
    \item ``Jackknife instrumental variables estimation'' Angrist, Imbens and Krueger (1999)
    \item ``Random effects estimators with many instrumental variables'' Chamberlain and Imbens (2004)
    \item ``Tolerating defiance? Local average treatment effects without monotonicity'' de Chaisemartin
    \item ``Weak Instruments in Instrumental Variables Regression: Theory and Practice'', Andrews, Stock and Sun (2018)
    \item ``Local instrumental variables and latent variable models
      for identifying and bounding treatment effects'' Heckman and
      Vytlacil (1999)
    \end{itemize}
  \item Instrumental Variables (Part III)
    \begin{itemize}
    \item ``On the structure of IV estimands'' Andrews (2019)
    \item ``Weak instruments in instrumental variables regression: Theory and practice'' Andrews, Stock and Sun (2019)
    \item ``Jackknife instrumental variables estimation'' Angrist, Imbens and Krueger (1999)
    \item ``Random effects estimators with many instrumental variables'' Chamberlain and Imbens (2004)
    \item ``Tolerating defiance? Local average treatment effects without monotonicity'' de Chaisemartin
    \item ``Weak Instruments in Instrumental Variables Regression: Theory and Practice'', Andrews, Stock and Sun (2018)
    \item ``Local instrumental variables and latent variable models
      for identifying and bounding treatment effects'' Heckman and
      Vytlacil (1999)
    \end{itemize}
  \item Bartik + Simulated Instruments
    \begin{itemize}
    \item ``Bartik Instruments: What, When, Why and How'' Goldsmith-Pinkham, Sorkin and Swift (2020)
    \item ``Quasi-experimental shift-share research designs'' Borusyak, Hull and Jaravel (2020)
    \item ``Shift-share designs: Theory and inference'' Adao, Kolesar and Morales (2019)
    \item ``Non-random exposure to exogenous shocks: Theory and applications'' Borusyak and Hull (2021)
    \item ``The Estimation of Treatment Effects in Simulated Instrument Designs'', Aronow, Goldsmith-Pinkham and Sorkin (mimeo) 
    \end{itemize}
  \item Examiner Designs aka Judge IV 
    \begin{itemize}
    \item ``Consumer bankruptcy and financial health'' Dobbie, Goldsmith-Pinkham and Yang (2016)
    \item ``The criminal and labor market impacts of incarceration.'' Mueller-Smith (2015)
    \item ``Judging Judge Fixed Effects'' Frandsen, Lefgren and Leslie (2020)
    \item ``Racial bias in bail decisions'' Arnold, Dobbie, and Yang (2018)
    \item ``Family Welfare Cultures'', Dahl, Kostol and Mogstad (2014)
    \end{itemize}
  \item Regression Discontinuity I: Identification and Groundwork
    \begin{itemize}
    \item ``Identification and estimation of treatment effects with a regression-discontinuity design'' Hahn, Todd and Van Der Klaauw (2001)
    \item ``A Practical Introduction to Regression Discontinuity Designs: Foundations'', Cattaneo, Idrobo and Titiunik, (2020)
    \item ``A Practical Introduction to Regression Discontinuity Designs: Extensions,''  Cattaneo, Idrobo and Titiunik, (2021)
    \item ``Inference in Regression Discontinuity Designs with a Discrete Running Variable'', Kolear and Rothe (2018)
    \item ``Why High-Order Polynomials Should Not Be Used in Regression Discontinuity Designs'', Gelman and Imbens (2018)
    \item ``Inference on causal effects in a generalized regression kink design'' Card et al. (2015)
    \item ``Robust nonparametric confidence intervals for regression-discontinuity designs'' Calanico et al. (2014)
    \item ``Regression discontinuity designs using covariates'' Calanico et al (2019)
    \end{itemize}
  \item Regression Discontinuity II: The Checklist
  \item Regression Discontinuity III: Extensions
    \begin{itemize}
    \item ``Approximate Permutation Tests and Induced Order Statistics in the Regression Discontinuity Design'', Canay and Kamat (2017)
    \item ``Manipulation of the running variable in the regression discontinuity design: A density test'', McCrary (2008)
    \item ``External Validity in Fuzzy Regression Discontinuity Designs'', Bertanha and Imbens (2019)
    \item ``Bounds on treatment effects in regression discontinuity designs with a manipulated running variable'', Gerard, Rokkanen and Rothe (2020)
    \item ``A Simple Adjustment for Bandwidth Snooping'', Armstrong and Kolesar (2018)
\end{itemize}     
  % \item Bunching Estimators
  %   \begin{itemize}
  %   \item ``On Bunching and Identification of the Taxable Income Elasticity'' Blomquist et al. (2019)
  %   \end{itemize}
  \end{enumerate}
\item \underline{Machine Learning}
  \begin{enumerate}
  \item Supervised Machine Learning I: Prediction
    \begin{itemize}
    \item ``Machine Learning Methods Economists Should Know About.''Athey S, Imbens G.  (2019)
    \item ``Predictably Unequal? The Impact of Machine Learning on Credit Markets'' Fuster et al. (2020)
    \item ``Deep Neural Networks for Estimation and Inference'' Ferrell, Liang and Misra (2020)
    \end{itemize}
  \item Supervised Machine Learning II: Heterogeneous Treatment Effects
    \begin{itemize}
    \item ``Estimation and Inference of Heterogeneous Treatment Effects using Random Forests'' Wager and Athey (2019)
    \item ``Recursive partitioning for heterogeneous causal effects'' Athey and Imebns (2016)
    \item ``Using Causal Forests to Predict Treatment Heterogeneity:  An Application to Summer Jobs'' Davis and Heller (2017)
    \item ``Generic Machine Learning Inference On Heterogenous
      Treatment Effects In Randomized Experiments, With An Application
      To Immunization In India'' Chernozhukov et al. (2020)
    \end{itemize}
  \item Unstructured Data and Unsupervised Machine Learning 
    \begin{itemize}
    \item ``Text as Data'', Kelly, Gentzkow and Taddy (2020)
    \item ``Measuring Technological Innovation Over the Long Run'', Kelly et al. (2020)
    \item ``Parsing the content of bank supervsision'', Goldsmith-Pinkham, Hirtle and Lucca (2017)
    \item ``Grouped Patterns of Heterogeneity in Panel Data'', Bonhomme and Manresa (2015)
    \item ``Computer Vision and Real Estate: Do Looks Matter and Do Incentives Determine Looks'', Glaeser, Kincaid, and Naik (2018)
    \item ``Word Power: A New Approach for Content Analysis'', Jegadeesh and Wu (2013)
    \item ``Text Selection'', Kelly, Manela and Moreira (2020)
    \item ``The Structure of Economic News'', Bybee et al. (2020)
    \end{itemize}
  \end{enumerate}
\item \underline{Miscellaneous}
  \begin{enumerate}
  \item Partial Identification
    \begin{itemize}
    \item ``Nonparametric bounds on treatment effects'' Manski. (1990)
    \item ``Confidence intervals for partially identified parameters'' Imbens and Manski (2004)
    \item ``Inference on regressions with interval data on a regressor or outcome'' Manski and Tamer (2002)
    \item ``Estimation and Confidence Regions for Parameter Sets in Econometric Models'' Chernozhukov, Hong and Tamer (2007)
    \item ``Training, Wages, and Sample Selection: Estimating Sharp Bounds on Treatment Effects'' Lee (2009)
    \item ``Better Lee Bounds'' Semenova (2020)      
    \item ``Monotone Treatment Response'' Manski (1997)
    \item ``Partial Identification in Econometrics'', Tamer (2010)
    \item ``Microeconometrics with Partial Identification'', Molinari (2020)
    \end{itemize}
  \end{enumerate}
\end{enumerate}

\vspace*{.15in}
\noindent\textbf{Grading Policy:} Grades will be based on the problem sets, divided evenly over the problem sets. There will be no mid-term or final exam

\vskip.15in
\noindent\textbf{Class Policy:}  
\begin{itemize}
\item Regular attendance is not required, due to Covid circumstances, but is preferred. 
\end{itemize}

%%%%%% THE END 
\end{document} 